\begin{abstract}
The heritability of human connectomes - the maps of brain connectivity - is crucial for understanding the influence of genetic and environmental factors on variability in connectomes and their implications for behavior and disease. Unfortunately, current heritability methods can mistake correlation for causation or use models that oversimplify the intricate nature of brain networks, which risks misinterpreting the influence of genetics on brain connectivity patterns. To address these limitations, we propose a new approach that combines causal inference techniques to control for potential confounding with statistical models designed to capture the complex dependencies within brain networks. This allows us to explore more nuanced concepts of connectome heritability by systematically removing shared network structures. Applying our methods to diffusion connectomes derived from the Human Connectome Project, we show that brain connectivity is heritable heritability, even after adjustments for factors like neuroanatomy, age, and sex. However, once we address for common network between connectomes, our causal tests are no longer significant. These results suggest that previous conclusions on connectome heritability may be driven by the shared network structures. Thus, this work highlights the importance for future works to continue to develop data-driven heritability models which faithfully reflect potential confounders and network structures.

% The heritability of human connectomes is crucial for understanding the influence of genetic and environmental factors on variability in connectomes, and their implications for behavior and disease. However, current methods for studying heritability assume an associational rather than a causal effect, or rely on strong distributional assumptions that may not be appropriate for complex, high-dimensional connectomes. To address these limitations, we propose two solutions: first, we formalize heritability as a causal problem to identify measured covariates and to control for unmeasured confounding, allowing us to make causal claims. Second, we leverage statistical models that capture the underlying structure and dependence within connectomes, enabling us to define different notions of connectome heritability by removing common network structures such as scaling of edge weights between connectomes. We then develop a non-parametric test to detect whether causal heritability exists after taking principled steps to adjust for these commonalities, and apply it to diffusion connectomes estimated from the Human Connectome Project. Our findings reveal that heritability can still be detected even after adjusting for potential confounding like neuroanatomy, age, and sex. However, once we address for rescaling between connectomes, our causal tests are no longer significant. These results suggest that previous conclusions on connectome heritability may be driven by rescaling factors. Together, our manuscript highlights the importance for future works to continue to develop data-driven heritability models which faithfully reflect potential confounders and network structure.


% The heritability of human connectomes is crucial for understanding the influence of genetic and environmental factors on variations in connectomes, and their implications for behavior and disease. Previous heritability studies of human connectomics typically operate under strong distributional assumptions, such as the edges are independent, and then search merely for differences across populations. Specifically, they are prone to claim that edges are heritable without address the potential confounds of scaling or neuroanatomy. We therefore developed causal methods based on semi-parametric models that that relax those assumptions. Via simulations, we illustrate how our methods are insensitive to variations such as  global rescalings of all edges in the connectome. Running these methods on the Human Connectome Project dataset, we find that once we address rescaling and measured potential confounders such as neuroanatomy, age, and sex, heritability is no longer detected at a global or local scale. We therefore argue that previous conclusions of heritability be re-investigated in light of these considerations, and future studies similarly address potential confounds in their data. 

% Understanding the heritability of human connectomes is essential for comprehending the impact of genetic and environmental factors on connectome variations, and their subsequent implications for behavior and disease. Traditional heritability studies in human connectomics often rely on strong distributional assumptions, such as the independence of edges, and primarily focus on detecting differences across populations. However, these approaches tend to claim heritability of edges without adequately addressing potential confounding factors like scaling or neuroanatomy. To address these limitations, we developed causal methods grounded in semi-parametric models that relax the aforementioned assumptions. Through simulations, we demonstrate that our methods remain robust to variations, such as global rescalings of all connectome edges. By applying these methods to the Human Connectome Project dataset and accounting for potential confounders like neuroanatomy, age, and sex, we discovered that heritability is no longer detectable at either global or local scales. In light of these findings, we argue that previous conclusions on heritability should be re-examined while considering these factors. Moreover, future studies should proactively address potential confounds in their data to provide a more accurate understanding of human connectome heritability.
\end{abstract} 

