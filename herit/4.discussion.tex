\section{Discussion} 
\subsection{Summary} \label{discussion:summary}
In this study, we presented the first examination of the heritability of human connectomes using causal modeling and statistical modeling of connectomes. Our investigations into the heritability of connectomes provides strong evidence that the structural connectivity patterns within them are highly influenced by genetics by leveraging causal models and developing a test procedure to detect heritability. To establish this, we first compared the connectomes of monozygotic twins with same-sex dizygotic twins, which allowed us to account for shared environmental factors. Our results demonstrated that not only are connectomes reliable and meaningful, but they are also significantly influenced by genetics. We then extended our analysis to include monozygotic, dizygotic, and non-twin siblings, which further confirmed that connectomes are heritable. However, these results may be confounded by both observed and unobserved covariates. Therefore, we define a set of covariates, which we believe can account for all possible other unobserved covariates. Given this, we found that connectomes remained heritable even after controlling for age and the mediating effect of neuroanatomy. Finally, we demonstrated that the subgraphs formed by sets of brain regions that are not heritable within connectomes are also heritable. This suggests that there is an underlying structure within these subgraphs that is not evident in individual vertices. 
%By uncovering the heritable nature of connectomes, our findings contribute to the broader understanding of genetic influences on brain connectivity and could potentially aid in uncovering new insights into the genetic basis of neurological disorders and cognitive traits.

A key innovation in this study is the utilization of statistical models, specifically the random dot product graphs, for connectomes, which introduces a new representation called latent positions. The models allow us to compare connectomes in various ways and explore different aspects of heritability. The simplest model we use is the exact model, which assesses any differences in the latent positions of connectomes. The global model, on the other hand, accounts for the possibility that connectomes may have different scales, such as variations in edge weights or more connections in general. Therefore, this model quantifies the differences in latent positions after removing the effect of global scaling. The vertex model allows for both global and vertex-wise scaling, enabling the three models to account for increasingly complex structures in connectomes as we move from the exact model to the vertex model. Our findings, as presented in Section \ref{sec:conditional_dcorr}, indicate that genetics play a crucial role in the differences of vertex scaling. However, when addressing vertex scaling of connectomes, heritability becomes undetectable, suggesting that heritability is only due to vertex scaling and not other underlying structures present in the connectomes. These results imply that previous conclusions on connectomic heritability may be influenced by vertex scaling and emphasize the need for further research into the relationship between vertex scaling in RDPG and network statistics, such as clustering coefficients and average path lengths.

There are several other open questions and opportunities for future research on the heritability of connectomes. One potential direction involves studying heritability in different neuroimaging modalities. Although our focus was on human connectomes estimated from diffusion MRI, the methods presented here can be applied to study heritability in connectomes obtained from other modalities, such as functional MRI, magnetoencephalography (MEG), and electroencephalography (EEG), or even in other species, like mice. Another potential direction is incorporating more detailed genetic data, including GWAS, to identify specific genetic variants associated with brain connectivity patterns. This would enable more direct and meaningful comparisons of genomes between subjects and could be readily integrated into the testing procedure presented here.

The findings on heritability in connectome analysis have significant implications for understanding the genetic basis of various neurological and psychiatric disorders. Disruptions in brain connectivity have been implicated in numerous conditions, such as autism spectrum disorder, schizophrenia, and Alzheimer's disease \cite{van2019cross, fornito2015connectomics}. By revealing the heritable components of brain connectivity, researchers can start to pinpoint specific genetic factors that contribute to the development of these disorders. This knowledge could inform the creation of novel diagnostic tools, treatment strategies, and targeted interventions, ultimately enhancing the prognosis and quality of life for individuals affected by these disorders.

% Our main contribution is introducing a novel way to conceptualize heritability as causal effects.  and proposing non-parametric tests to quantify it. Moreover, we employ statistical models for connectomes that introduce a new representation of connectomes called latent positions. This representation enables us to compare connectomes in different ways, resulting in various notions of heritability. The simplest model we use is the exact model, which assesses any differences in the latent positions of connectomes. The global model, on the other hand, accounts for the possibility that connectomes may have different scales, such as variations in edge weights or more connections in general. Therefore, this model quantifies the differences in latent positions after removing the effect of global scaling. The vertex model allows for both global and vertex-wise scaling, enabling the three models to account for increasingly complex structures in connectomes as we move from the exact model to the vertex model.

\subsection{Limitations} \label{discussion:limitations}
\subsubsection{Estimating Heritability}
Heritability is typically defined as the relative contribution of genetic variations as opposed to environmental variations in phenotypic traits, and tests whether the heritability estimate is statistically significant, meaning if heritability exists or not. Numerous methods have been developed to partition phenotypic variation into genetic and environmental components, including techniques such as structural equation modeling (SEM). However, the causal approach presented in this study is distinct from these methods, as it does not aim to estimate heritability in a quantitative manner. Instead, our approach focuses on identifying the presence or absence of heritability by examining causal relationships between genetic factors and phenotypic traits.

The causal approach presented in this study offers a complementary perspective that can contribute to resolving the missing heritability problem. The missing heritability problem arises when the estimated heritability from genetic markers, such as those identified in genome-wide association studies (GWAS), accounts for only a small fraction of the heritability inferred from twin studies using SEMs or similar methods. This discrepancy between the observed and expected heritability is a major challenge in genetic research, as it limits our understanding of the genetic architecture underlying complex traits and diseases. By focusing on the presence or absence of causal relationships between genetic factors and phenotypic traits, this approach can help to identify previously unexplored genetic influences that might not be captured by traditional heritability estimation methods.

\subsubsection{Choice of Model}
We utilized the random dot product graph ($\rdpg$) model in our analysis, which provided a unified framework for analyzing connectomes at both the network and individual vertex levels. However, it is important to acknowledge that every model has its limitations. The $\rdpg$\ model can only represent a subset of all possible stochastic block models, and may not be able to capture certain patterns in connectivity. For example, if human brains were characterized by more connections between hemispheres than within hemispheres, the $\rdpg$\ model will not be able to capture these patterns. To address these limitations, other models have been developed, such as the generalized random dot product graph, which can represent a broader range of stochastic block models and other network models. Additionally, there are models that are designed to analyze populations of networks, such as the common subspace independent edge model and joint random dot product graph model. Choosing any of these models enables one to define a valid distance metric for comparing connectomes. It is worth noting that there is currently no consensus on which model is best suited for modeling the human brain. Therefore, researchers must carefully consider the advantages and limitations of each model when selecting the appropriate one for their analysis. Ultimately, the choice of model will depend on the research question, the data available, and the assumptions that are deemed reasonable for the specific context.