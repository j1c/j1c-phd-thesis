\chap{Abstract} 
%%%% your abstract goes here (word limit: 350)
Neuroscientists have made tremendous progress in our ability to measure the structure
of neural circuits. Detailed maps of neural wiring – termed connectomes – are
increasingly studied in neuroscience because they have the potential to improve our
understanding of how structure relates to function, how the structure of the brain
is generated, how it changes with evolution or disease or experience. Despite this
progress in measuring connectomes at the scale of individual neurons and the synapses
between them, techniques for extracting meaning from these complicated datasets
have lagged behind.
This thesis focuses on developing methods for improving our understanding of
connectomics data, with a focus on the application of these methods to the connectome
of a Drosophila larva brain. In particular, this thesis describes methods for characterizing the general structure of a connectome, including analysis of connectivity-based
cell types, characterization of feedforward and feedback structure in a biological
neural network, and tools for assessing sensory convergence within the brain using
network traversal methods. Then, this thesis presents novel methods for comparing
and aligning connectome datasets via a focus on the comparison of the left and right
hemispheres of this Drosophila larva brain. The first method enables a statistical
comparison of cell type connection probabilities in the left and the right hemispheres,
enabling quantitative assessment of which parts of the two connectome datasets have
the most significant deviations. The second method enables high-accuracy automated
prediction of neuron-to-neuron pairings across the two sides of this connectome by
augmenting techniques for graph matching. The tools used and developed as part of
this thesis are also made available to the wider neuroscience community and beyond,
via the development of documented, tested, open-source Python code to implement
these algorithms.
Taken together, this thesis represents an advancement in the algorithmic analysis
of connectome data. This kind of analysis will be particularly important going forward,
as larger and more complicated connectomes are generated, and in particular, when
multiple connectome samples are collected which require quantitative comparison



%%%%  committee members (add it right after the abstract w/o page break)
\begin{singlespace}

\section*{Primary reader and thesis advisor:}

Dr. Joshua T. Vogelstein (Primary Advisor)\\
Associate Professor \\
Department of Biomedical Engineering\\
Johns Hopkins University, Baltimore, MD 

%% if you have co-advisor, add here w/ \vspace{0.1in} as shown below

\section*{Secondary readers:}

Dr. Carey Priebe \\
Professor \\
Department of Applied Mathematics and Statistics\\
Johns Hopkins University, Baltimore, MD 

\vspace{0.1in}

Dr. Michael Powell \\
Assistant Professor \\
Department of Mathematical Sciences\\
United States Military Academy, West Point, NY

%%%% add more readers if you have on your committee 
%% use \vspace{0.1in} in between members for clarity
%% alternatively you can use minipage environment to put side-by-side

\end{singlespace}