\chap{Abstract} 

Advances in \textit{in vivo} brain imaging techniques have empowered neuroscientists to generate extensive datasets characterizing brains. Detailed maps of neural connectivity – termed connectomes - offer two fundamental perspectives: structural connectomes chart the physical wiring of the brain, while functional connectomes map patterns of correlated activity across brain regions. Investigating this diverse population of connectomes is vital for deciphering the principles guiding brain development, the neurological changes associated with disease, how our brains are shaped by experience, and the evolutionary trajectory of the human brain. Despite the progress in measuring connectomes at the scale at large, techniques for extracting meaning from these complicated datasets have lagged behind.

This thesis develops novel methods to enhance our understanding of connectome populations, with a focus on human structural and functional networks.  It begins by reviewing existing statistical models and algorithms for analyzing connectomes and networks as well as improvements to existing algorithms for comparing connectomes. Next, it presents novel methods for investigating the heritability of structural connectomes within a causal analysis framework, while leveraging statistical modeling for connectomes. Additionally, we introduce a new method to quantify temporal dependencies within functional connectomes using generalized correlation measures. To foster broader research impact, these tools are made accessible to the neuroscience community and beyond through a documented, tested, open-source Python package.

Collectively, this thesis advances the algorithmic analysis of connectome data, and this work brings us closer to a better understanding of the brain's intricate workings. As the field of connectomics expands, generating increasingly complex and large-scale datasets, these sophisticated analytical methods will be essential.


%%%%  committee members (add it right after the abstract w/o page break)
\begin{singlespace}

\section*{Thesis Readers}

Dr. Joshua T. Vogelstein (Primary Advisor)\\
Associate Professor \\
Department of Biomedical Engineering\\
Johns Hopkins University, Baltimore, MD 

\vspace{0.1in}

Dr. Carey E. Priebe \\
Professor \\
Department of Applied Mathematics and Statistics\\
Johns Hopkins University, Baltimore, MD 

\vspace{0.1in}

Dr. Michael Powell \\
Assistant Professor \\
Department of Mathematical Sciences\\
United States Military Academy, West Point, NY

%%%% add more readers if you have on your committee 
%% use \vspace{0.1in} in between members for clarity
%% alternatively you can use minipage environment to put side-by-side

\end{singlespace}