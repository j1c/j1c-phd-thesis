\chapter{Introduction} \label{chap-intro}

The field of connectomics stands at the forefront of a revolution in neuroscience, promising to unravel how the brain gives rise to thought, emotion, and behavior. At its core, connectomics seeks to map the comprehensive network of connections within the brain – the connectome. The connectome encompasses both structural connections, the physical ``wiring'' of neurons, and functional connections, the patterns of correlated activity that emerge across brain regions, which are often represented as networks.

Networks offer a powerful framework for representing and analyzing the intricate relationships within the brain. In a connectome, vertices represent brain regions, while edges are their structural connections (such as white matter tracts) or functional links (based on correlations in brain activity). Network analysis techniques can reveal key hubs, modules of highly interconnected regions, and pathways shaping communication within the brain \cite{sporns2005human, behrens2012human, van2016comparative, fornito2015connectomics, griffa2013structural}. However, analyzing network data like connectomes poses challenges that traditional machine learning methods often struggle to address. Statistical network models provide a specialized toolkit, allowing us to explore the generative processes behind networks and make principled comparisons. Importantly, interpreting findings from these models requires careful consideration of their underlying assumptions. In the first parts of the thesis, Chapters \ref{chap:statistical} and \ref{chap:nonpar} delve into an overview of existing statistical network models, algorithms, and their applications to analysis of connectomics data. This foundation is crucial, as tools designed specifically for network data, including the challenges of noise and uncertainty, are essential for unlocking the full potential of connectomics research.

The study of heritability in neuroscience boasts a rich history, dating back to the pioneering work of researchers like Francis Galton in the 19th century. Heritability refers to the proportion of observed variation in a particular brain-related trait (such as intelligence, susceptibility to disease, or brain connectivity patterns) that can be attributed to genetic factors. Understanding the heritability of brain connectivity can pave the way for identifying genetic risk factors, developing targeted interventions, and ultimately, promoting personalized approaches to mental health and neurological care. Using statistical networks models from Chapter \ref{chap:statistical} along with causal modeling of variables that affect both genomes and connectomes, we investigate the heritability of human structural connectomes in Chapter \ref{chap:heritability}. This work enables further investigations in connectomic heritability in other modalities such as functional connectomes from functional MRI (fMRI) or from electroencephalogram (EEG). 

Lastly, it is crucial to make the computation and network analysis tools developed within this thesis widely available to foster collaboration and accelerate discoveries. Chapter \ref{chap:m2g} introduces \texttt{m2g}, an open-source pipeline for estimating structural connectomes from diffusion magnetic resonance images (dMRI), which was used to estimate the structural connectomes for studying heritability in Chapter \ref{chap:heritability}. Chapter \ref{chap:graspologic} introduces \texttt{graspologic}, an open-source Python software package which implemented these methods. These packages, along with its comprehensive documentation, tutorials, and support from the developers, empowers the neuroscience community and other researchers to leverage these techniques, driving advancements in our understanding of brain connectivity and its related fields.

