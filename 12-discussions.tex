\chapter{Discussion}\label{chap:discussion}

This thesis presented novel methods for extracting principles from brain connectivity data, and demonstrated the application of these methods on structural connectomes of human brains. Despite these advances, many avenues remain for expanding upon this work and developing methods which will enable further biologically relevant inferences from connectome data.

\section{Heritability of Functional Connectomes}
Functional connectomes also play a critical role in the diagnosis and treatment of neuropsychiatric disorders. Abnormalities in brain connectivity have been implicated in a wide range of conditions, including schizophrenia, depression, autism, and Alzheimer’s disease. By studying the functional connectomes of individuals with these disorders, researchers can identify characteristic connectivity patterns that may serve as biomarkers for early diagnosis. Furthermore, this knowledge can inform the development of targeted interventions, such as neurostimulation therapies and personalized medicine approaches, aimed at restoring normal connectivity and improving clinical outcomes.
